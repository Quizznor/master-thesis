% !TEX root = ../thesis.tex

\chapter{Classical station triggers}
\label{chap:classical-triggers}

As mentioned in \autoref{chap:auger-observatory}, continously analyzing data sent to CDAS from each of the 1600 SD water tanks would quickly exceed the 
computational capabilites of Augers' main servers. For this purpose, trace information is only collected from a station, once a nearby T3 event 
(c.f. \autoref{sssec:t3-trigger}) has been detected. The formation of a T3 trigger is dependant on several T2, or station-level, triggers, which will
be discussed in this chapter. First, the implementation of different trigger algorithms is discussed in \autoref{sec:classical-trigger-implementation}.
Their performance is evaluated in \autoref{sec:classical-triggers-performance}.

\section{Implementation}
\label{sec:classical-triggers-implementation}

\subsection{Threshold trigger (Th)}
\label{ssec:threshold-trigger}

\subsection{Time over threshold trigger (Tot)}
\label{ssec:time-over-threshold-trigger}

\subsection{Time over threshold deconvoluted (Totd)}
\label{ssec:time-over-threshold-deconvoluted}

\subsection{Multiplicity of positive steps (Mops)}
\label{ssec:multiplicity-of-positive-steps}



\section{Performance}
\label{sec:classical-triggers-performance}




