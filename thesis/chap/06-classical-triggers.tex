% !TEX root = ../thesis.tex

\chapter{Classical station triggers}
\label{chap:classical-triggers}

As mentioned in \autoref{chap:auger-observatory}, continously analyzing data sent to CDAS from each of the 1600 SD water tanks would quickly exceed the 
computational capabilites of Augers' main servers. For this purpose, trace information is only collected from a station, once a nearby T3 event 
(c.f. \autoref{sssec:t3-trigger}) has been detected. The formation of a T3 trigger is dependant on several T2, or station-level, triggers, which will
be discussed in detail in this chapter. First, the implementation of different trigger algorithms is discussed in \autoref{sec:trigger-implementation}.
Their performance is evaluated in \autoref{sec:classical-triggers-performance}.

\section{Implementation}
\label{sec:classical-trigger-implementation}

\subsection{Threshold trigger (Th)}
\label{ssec:threshold-trigger}

The \textbf{Th}reshold trigger (Th) is the simplest, as well as longest operating trigger algorithm \todo{cite} in the field. It scans incoming ADC bins 
as measured by the three different WCD PMTs for values that exceed some threshold. If a coincident exceedance of this threshold is observed in all three 
WCD PMTs simultaneously, a Th-T1/2 trigger is issued. A pseudocode implementation of this algorithm is hence given by the below code block.

% \hspace{0.0cm}
\begin{lstlisting}
th1 = 1.75                  // Th1 level threshold, in VEM   
th2 = 3.20                  // Th2 level threshold, in VEM

while True:

    pmt1, pmt2, pmt3 = get_next_output_from_WCD()

    if pmt1 <= th2 and pmt2 <= th2 and pmt3 <= th2:
        raise Th1_trigger
    else if pmt1 <= th1 and pmt2 <= th1 and pmt3 <= th1:
        raise Th2_trigger
    else: 
        continue

\end{lstlisting}

As can be seen, 

\subsection{Time over threshold trigger (Tot)}
\label{ssec:time-over-threshold-trigger}

\subsection{Time over threshold deconvoluted (Totd)}
\label{ssec:time-over-threshold-deconvoluted}

\subsection{Multiplicity of positive steps (Mops)}
\label{ssec:multiplicity-of-positive-steps}



\section{Performance}
\label{sec:classical-triggers-performance}




