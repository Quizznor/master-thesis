% !TEX root = ../thesis.tex

\chapter{Neural network training data}
\label{chap:neural-network-data}

Over their relatively brief existance, neural networks have been shown to perform increasingly impressive tasks (e.g.  \cite{openai2019dota}, \cite{openai2023gpt4}, 
and many more). However, they learn by example. The performance of a neural network is directly linked to the input data it receives during training. If the 
training data is not an accurate example of real world information a network later operates on, insight gained from it is at best an approximation, and at worst
completely randomly generated data. 

As such, it is not a question \textit{if} some neural network architecture can learn to identify an extensive air shower from WCD data, but rather which 
implementation, fed with which information, does. For this purpose, this chapter explains the procedure with which training data is generated. As stated above, this
must occur with a focus on being representative of data actually measured in the SD array. The elected approach to create time traces is modularized. The structure
of this chapter reflects this. First, general comments about the characteristics of background data (i.e. the WCD detector response in the absence of an extensive 
air shower) are made in \autoref{sec:background-dataset}. Next, the process to extract signal originating from CRs is detailed in \autoref{sec:signal-dataset}.
Lastly, building the time trace from the aforementioned modules and drawing samples from it for a neural network to train on is done in 
\autoref{sec:sliding-window-analysis}.

\section{Background dataset}
\label{sec:background-dataset}

\subsection{Random traces}
\label{ssec:random-traces}



\subsection{Accidental muons}
\label{ssec:accidental-muons}



\section{Signal dataset}
\label{sec:signal-dataset}


\section{Trace building \& Sliding window analysis}
\label{sec:sliding-window-analysis}