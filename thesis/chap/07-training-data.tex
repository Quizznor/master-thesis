% !TEX root = ../thesis.tex

\chapter{Neural network training data}
\label{chap:neural-network-data}

Over their relatively brief existance, neural networks have been shown to perform increasingly impressive tasks (e.g.  \cite{openai2019dota}, \cite{openai2023gpt4}, 
and many more). However, they learn by example. The performance of a neural network is directly linked to the input data it receives during training. If the 
training data is not an accurate example of real world information a network later operates on, insight gained from it is at best an approximation, and at worst
completely randomly generated data. 

As such, it is not a question \textit{if} some neural network architecture can learn to identify an extensive air shower from WCD data, but rather which 
implementation, fed with which information, does. For this purpose, this chapter explains the procedure with which training data is generated. As stated above, this
must occur with a focus on being representative of data actually measured in the SD array. The elected approach to create time traces is modularized. The structure
of this chapter reflects this. First, general comments about the characteristics of background data (i.e. the WCD detector response in the absence of an extensive 
air shower) are made in \autoref{sec:background-dataset}. Next, the process to extract signal originating from CRs is detailed in \autoref{sec:signal-dataset}.
Lastly, building the time trace from the aforementioned modules and drawing samples from it for a neural network to train on is done in 
\autoref{sec:sliding-window-analysis}.

\section{Background dataset}
\label{sec:background-dataset}

While a flux of partices causes elevated ADC levels in both the HG and LG channels of a WCD PMT during a shower event, the lack of such a phenomenon does not imply 
the readout information is uniformly flat. Instead, it hovers around the channels' baseline (c.f. \autoref{sec:surface-detector}) with occasional spikes upwards 
due to low-energy particles impinging on the detector. Coupled with electronic noise from the many digital components in the UUB, this constitutes the data that is 
collected inbetween air shower events.

\subsection{Accidental muons}
\label{ssec:accidental-muons}

Most low-energy background particles present in the detector are muons. These are produced in the upper atmosphere during cascading processes analog to 
\autoref{chap:physical-background}. Due to the low primary energy the electromagnetic component of the shower is thermalized before it reaches surface level. The 
muonic component by itself does not contain enough information to enable an accurate reconstruction of primary energy and origin. This fact, coupled with the high
flux of events at lower energies ($\Upphi|_{E=\SI{100}{\giga\electronvolt}} \approx \SI[per-mode=power]{1}{\per\meter\per\second}$ \cite{boezio2000measurement}) 
make these events unsuitable for analysis. Stray muons, even though they originate from extensive air showers, must consequently be considered background events.

The rate at which such particles traverse a WCD tank is $f_\text{Acc.}\approx\SI{4.665}{\kilo\hertz}$ \cite{DavidInjectionFrequency}. Their arrival time is 
Poisson-distributed. This implies that generally, one in 14 time traces contains signal from a low-energy background event. Coincidences of two accidental 
muons occur on a sub-percent level. Any higher order of coincidences is likely originating from a single air shower process.

\subsection{Electronic noise}
\label{ssec:electronic-noise}

Electronic noise is the umbrella term given to the distortions that some signal is subject to from creation to digital readout. In the electronics of the SD array,
this noise is assumed to be Gaussian. \todo{RMS monitoring}

\subsection{Random traces}
\label{ssec:random-traces}


\section{Signal dataset}
\label{sec:signal-dataset}



\section{Trace building \& Sliding window analysis}
\label{sec:sliding-window-analysis}
