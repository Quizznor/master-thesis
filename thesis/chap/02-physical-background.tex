%! TEX root = ../thesis.tex

\chapter{Physical background}
\label{chap:physical-background}

This chapter aims to introduce the general physical principles underlaying the analysis presented in this work. For this purpose, an overview of 
the origin, composition and energy spectrum of cosmic rays is given in \autoref{sec:cosmic-rays}. Their interactions with other matter, and
consequently possible detection methods are listed in \autoref{sec:extensive-air-showers}.

\section{Cosmic rays}
\label{sec:cosmic-rays}

\subsection{History}
\label{ssec:cr-history}

A first hint at the existance of high-energy particles in the upper atmosphere was given by Hess in 1912, who found that the discharge rate of 
an electroscope is altitude-dependant. Millikan coined the term cosmic "rays" for these particles, as he argued the ionizing radiation must be 
part of the electromagnetic spectrum \cite{millikan1928origin}. This was later - at least partially - falsified with the discovery of the 
east-west effect \cite{johnson1938note}. Hess' observation however withstood the tests of time and  was ultimately recognized with the Nobel 
prize in physics in 1936 \cite{nobelprize1936}. Two years later, in 1938, Pierre Auger showed via coincidence measurements that cosmic rays 
originate from outer space, and gave a first description of extensive air showers. Another 60 years later, the Pierre Auger collaboration would 
adopt his experimental setup and name in their search for cosmic rays of the highest energies.

Numerous other discoveries have helped advance our knowledge in both astro- and particle physics in the meantime. These include (but are not 
limited to) 

... \TODO

\subsection{Origin}
\label{ssec:cr-origin}

Cosmic rays whose kinetic energy far exceeds their rest energy must originate from some of the most extreme environments in space. In particular,
regions with a large (either in field strength or spatial extent) electromagnetic field, where charged particles are accelerated to speeds very
close to the speed of light, \speedoflight, via e.g. the Lorentz force.

... \TODO

\subsubsection{Acceleration mechanisms}
\label{sssec:cr-acceleration-mechanisms}



\subsection{Composition}
\label{ssec:cr-composition}

\subsection{Energy spectrum}
\label{ssec:cr-energy-spectrum}


\section{Extensive air showers}
\label{sec:extensive-air-showers}

Consider an incident particle of sufficiently high energy such that 

\subsection{Heitler Model}
\label{ssec:heitler-model}

\subsection{Heitler-Matthews Model}
\label{ssec:heitler-matthews-model}








