%! TEX root = ../thesis.tex

\chapter{Physics of cosmic rays}
\label{chap:physical-background}

This chapter aims to introduce the general physical principles underlaying the analysis presented in this work. For this purpose, an overview of the origin, 
composition and energy spectrum of cosmic rays is given in \autoref{ssec:cr-history}, \autoref{sec:cr-origin-and-composition}, and \autoref{sec:cr-energy-spectrum}
respectively. Their interactions with other matter, the physics of extensive air showers and their possible detection methods are listed in 
\autoref{sec:extensive-air-showers}.

\section{History}
\label{ssec:cr-history}

A first hint at the existance of high-energy particles in the upper atmosphere was given by Hess in 1912, who found that the discharge rate of an electroscope is 
altitude-dependant. Millikan coined the term cosmic "rays" for these particles, as he argued the ionizing radiation must be part of the electromagnetic spectrum 
\cite{millikan1928origin}. This was later, at least partially, falsified with the discovery of the east-west effect \cite{johnson1938note}. Hess' observation 
however withstood the tests of time and was ultimately recognized with the Nobel prize in physics in 1936 \cite{nobelprize1936}. Two years later, in 1938, Pierre 
Auger showed via coincidence measurements that cosmic rays in fact originate from outer space, and gave a first description of extensive air showers 
\cite{auger1939extensive}. Another 60 years later, the Pierre Auger collaboration would adopt his experimental setup and name in their search for cosmic rays of 
the highest energies.

In the meantime, numerous results from different cosmic ray detectors all over the globe have helped propel the related fields of particle physics, astro physics 
and cosmology to new insights. Observations from cosmic ray physics serve as a valuable cross-check to the hadronic interaction models developed e.g. at CERN 
\cite{ostapchenko2007status}. New theories modeling the final moments in the life of stars have arisen thanks to results from e.g. Kamiokande 
\cite{goldman1988implications}. 


\section{Origin \& Composition}
\label{sec:cr-origin-and-composition}

Cosmic rays whose kinetic energy far exceeds their rest energy must originate from some of the most extreme environments in space. In particular,
regions with a large (either in field strength or spatial extent) electromagnetic field, where charged particles are accelerated to speeds very
close to the speed of light, $c$, via e.g. the Lorentz force.

\section{Energy spectrum}
\label{sec:cr-energy-spectrum}


\section{Extensive air showers}
\label{sec:extensive-air-showers}








