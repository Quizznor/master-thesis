% !TEX root = ../thesis.tex

\chapter*{Abstract}

This thesis tests the potential of station-level triggers for the surface detector of the Pierre Auger Observatory. With the development of an extensive python 
framework to create realistic WCD time traces, a novel approach for the development of classification algorithms based on concepts of machine learning becomes 
possible. An in-depth discussion and evaluation of several candidate algorithms reveals the potential of neural networks in detecting extensive air showers. One 
particularly promising trigger are LSTM networks. Using a multi-layer instance of this type of recurrent neural network, and training it with a charge cut of 
$t_S = \SI{0.5}{\Charge}$, the overall efficiency of station-level triggers can be increase from $\sim8\%$ to $\sim18\%$ for primary particles of energies 
$\SI{10}{\peta\electronvolt} < E \leq \SI{32}{\exa\electronvolt}$. This results in T3 efficiency gains of up to $15.7$ percentage points at the highest examined
zenith angles.