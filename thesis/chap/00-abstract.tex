% !TEX root = ../thesis.tex

\chapter*{Abstract}

This thesis investigates the potential of station-level triggers for the surface detector of the Pierre Auger Observatory. The development of an extensive software
framework to generate realistic WCD time traces from measured background and simulated air showers enables the development of novel classification algorithms based
on concepts of machine learning. An in-depth evaluation of several candidate algorithms demonstrates the particular promise of LSTM networks, whose performance 
exceeds those of classical triggers. A multi-layer instance of this LSTM type neural network increases the efficiency of station-level triggers from $\sim8\%$ to 
$\sim18\%$ for triggering on stations with incident secondaries for proton primaries ranging from \SI{10}{\peta\electronvolt} to \SI{32}{\exa\electronvolt}. This 
results in gains in event detection efficiency of up to 15.7 percentage points at the highest examined zenith angles of $65^\circ$. The gains in efficiency with 
LSTMs were achieved without increasing the rate of background triggers, and only using 44 parameters. As such, LSTMs appear to be viable candidates for improving 
the trigger efficiencies of Water-Cherenkov detectors to extensive air showers without increasing demands on bandwidth between individual stations and central data
acquisition, and without requiring more powerful FPGAs than those used in modern air shower arrays. 